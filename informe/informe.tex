\documentclass[final,inline,a4paper,narroweqnarray]{ieee}
% In order to use the figure-defining commands in ieeefig.sty...
\usepackage{ieeefig}
% To use utf8 encoding
\usepackage[T1]{fontenc}
\usepackage[utf8]{inputenc}
\usepackage[spanish]{babel}
% to place figures
\usepackage{float}
\usepackage{graphicx}
\usepackage{amsmath}
% dont use geometry package
%\usepackage{geometry}

% "hack" to place figures in subsection
\usepackage{placeins}
\let\Oldsection\section
\renewcommand{\section}{\FloatBarrier\Oldsection}
\let\Oldsubsection\subsection
\renewcommand{\subsection}{\FloatBarrier\Oldsubsection}
\let\Oldsubsubsection\subsubsection
\renewcommand{\subsubsection}{\FloatBarrier\Oldsubsubsection}

\begin{document}

%----------------------------------------------------------------------
% Title Information, Abstract and Keywords
%----------------------------------------------------------------------
\title[TP N$^o$ 2: Rutas en Internet]{%
Trabajo Práctico N$^o$ 2: Rutas en Internet}

% format author this way for journal articles.
\author[Barrios, Benegas, Caravario, Rodriguez]{%
	Leandro Ezequiel Barrios,
	\and
	Gonzalo Benegas,
	\and
	Martin Caravario,
	\and
	Pedro Rodriguez
}

% make the title
\maketitle

% do the abstract
\begin{abstract}

En el presente Trabajo Práctico nos propondremos experimentar con herramientas 
y técnicas frecuentemente utilizadas a nivel de red. En particular, nos centraremos en la utilización de \texttt{traceroute} como herramienta para medir tiempos y \texttt{RTTS}. 

\end{abstract}

% start the main text ...

%----------------------------------------------------------------------
% SECTION I: Introduction
%----------------------------------------------------------------------
\section{Introducción}

Los objetivos del presente Trabajo Práctico son múltipes: por un lado, entender 
los protocolos involucrados en cada comunicación o intercambio de datos entre un 
usuario y un servidor, estando ambos conectados a Internet. Vimos en clase que 
estos protocolos son tanto de nivel de capa de transporte (TCP, UDP) como de 
aplicación (DNS, SMTP). Por otro lado, para afianzar nuestros conocimientos, 
implementaremos una herramienta que cumpla la misma función que 
\texttt{traceroute} por nuestra cuenta. La función de esta herramienta es 
la de detectar, a partir del nombre de un dominio (host), cada uno de los 
nodos intermedios por los que pasa el paquete antes de llegar a la dirección
 IP correspondiente a dicho host. El objetivo de esto será el de analizar los
 RTT a distintas universidades del mundo. 


%----------------------------------------------------------------------
% SECTION II: Caracterizando Rutas
%----------------------------------------------------------------------
\section{Caracterizando Rutas}

  %----------------------------------------------------------------------
  %  SUBSECTION III: Implementación de tool para hacer traceroute
  %----------------------------------------------------------------------
  \subsection{Implementación de tool para hacer traceroute}
  Una herramienta extra que implementamos, que nos facilitó la experimentación
  fue la implementación de una función, que llamamos DNSInverso, que para cada 
  IP nos permite reconocer, para cada router, su nombre.
  De esta forma podemos detectar, para cada máquina de la que obtenemos respuesta,
  a qué organización pertenece (si es que pertenece a alguna). Y, en particular,
  si pertenece a la universidad a la que estamos haciendo el traceroute.

  %----------------------------------------------------------------------
  %  SUBSECTION III: Adaptación de la tool
  %----------------------------------------------------------------------
  \subsection{Adaptación de la tool}
  Después de haber experimentado con la tool original, enviando paquetes
  echo-request a varias universidades alrededor del mundo, 
  adaptamos dicha \emph{tool} para que, una vez terminada 
  la búsqueda, calcule el \emph{valor standard}  o \emph{valor Z (ZRTT) } 
  en cada salto con respecto a la ruta global de la siguiente manera: si $RTT_i$ 
  es el RTT medido para el salto $i$, se define $ ZRTT_i = \dfrac{RTT_i 
  - \overline{RTT}}{SRTT} $ donde $\overline{RTT}$ y $SRTT$ son el promedio 
  y el desvío estandard de los RTTs de la ruta respectivamente.  

  Lo interesante de esta adaptación es que ...

%----------------------------------------------------------------------
% SECTION III: Traceroute a universidades del mundo
%----------------------------------------------------------------------
\section{Experimentación: Traceroute a universidades del mundo}

A partir de la herramienta \texttt{traceroute2} y tomando como IP origen la de la PC de uno de los integrantes del grupo,
realizamos rastreos para distintas universidades alrededor del mundo. Cada una de estas universidades las identificamos
a partir de su URL correspondiente. 
En el proceso de la experimentación, introdujimos los datos obtenidos para cada universidad en un graficador, que 
dada la lista de direcciones IP recorridas, las localiza geográficamente en un mapa del planeta Tierra, 
y las une mediante una línea. 

De esta forma, intentamos expresar de la forma más gráfica posible el camino a 
escala global que debió efectuar cada paquete para llegar desde nuestra PC hasta algún servidor de dicha universidad, 
ubicada a decenas de miles de kilómetros de nosotros. 

El problema básico consistió en encontrar universidades para las cuáles los trayectos efectuados por los paquetes 
tuvieran distintas características topológicas (por ejemplo que los paquetes atraviesen enlaces submarinos o que
las universidades queden en distintos continentes), a fin 
de poder realizar posteriormente un análisis más completo e interesante de los RTTs 
correspondientes a cada trayectoria.  

Dos de los problemas con los que nos encontramos fueron:
\begin{itemize}
	\item encontramos muchas universidades, como por ejemplo la de Nairobi (www.uombi.ac.ke), 
	para las cuáles el camino graficado en el mapa terminaba en EEUU, en lugar de de en África, 
	como esperábamos. Para estos
	casos, supusimos que el problema no era de un bug en nuestra implementación, sino en que
	dicha página web estaba hosteada
	en un servidor localizado en EEUU. Una posible razón por la cuál dicha universidad 
	querría hacer esto podría ser, por 
	ejemplo, que el costo de hostear un servidor en Kenia es mayor que en EEUU.
	\item en todos los casos analizados, hubieron muchos nodos intermedios que no 
	contestaban nuestro \emph{ICMP Request}, con
	lo cuál habían muchos \emph{hops} que no éramos capaces de localizar en el mapa, 
	lo cuál restaba precisión a nuestra 
	representación de la trayectoria de los paquetes.
	\item hubieron algunas universidades para las cuáles (por ejemplo, al enviar
	40 paquetes a www.udsm.ac.tz) había un primer router localizado en dicha universidad  
	nos contestaba, pero
	después de obtener esta respuesta, dejábamos de obtener respuesta. De esta forma, no 
	logramos obtener una respuesta de la IP destino. Creemos quue este 
	último router que nos contestaba estaba actuando como firewall de la red interna
	de dicha universidad.
	\item tuvimos problemas con el envío de paquetes a través de conexión wi-fi. Es por 
	esto que tuvimos que realizar todas las mediciones con una misma computadora, 
	conectada a internet a través de una conexión cableada.
\end{itemize}

La universidades que terminamos eligiendo fueron: en Noruega a <<www.uio.no>>, en Kazakstan a
 <<www.enu.kz>>, en Varsovia a <<www.uw.edu.pl>> y en Tokyo a <<www.u-tokyo.ac.jp>>.

\subsection{Gráficos y análisis}

%----------------------------------------------------------------------
% SECTION IV: Contrastando con la realidad
%----------------------------------------------------------------------
\section{Contrastando con la realidad}
A continuación simulamos una conexión mediante sucesivos \emph{pings} (\emph{EchoRequest} $+$ \emph{EchoReply}) con cada una de las universidades analizadas, y estimamos el \emph{RTT} en base a la siguiente ecuación: $ EstimatedRTT = \alpha * EstimatedRTT + (1 - \alpha) * SampleRTT $. 

\end{document}
